
% ====
\begin{rSection}{Experience}
{\bf AI Technology Center Intern, NVIDIA, HKSTP, Hong Kong}               \hfill { Nov.~2019 -- Present} \\
% \list{1}{2}
\textit{Routing Tree Construction \textbf{[{ASPDAC'21}]}}
\begin{itemize}[noitemsep,topsep=-5pt]
    \item Formalized special properties of the point cloud for the routing tree construction with theoretical proof.
    \item Outperformed previous methods (ISPD'18 best paper \& ICCAD'17 best paper) by a large margin yet being extensible and flexible.
\end{itemize}
\textit{Adaptive Layout Decomposition \textbf{[{DAC'20}]}}
\begin{itemize}[noitemsep,topsep=-5pt]
    \item Proposed an adaptive workflow for efficient decomposer selection and graph matching using graph embeddings.
    \item Reduced the runtime by 87.7\% while still preserving the optimality compared with the ILP-based decomposer.
\end{itemize}
\textit{Reviewed paper for CVPR'21}\\ \\
{\bf Research Assistant, The Chinese University of Hong Kong, NT, Hong Kong}               \hfill { Feb.~2019 -- July.~2019} \\
\textit{Open-source Layout Decomposition Framework \textbf{[{TCAD'21}]}}
\begin{itemize}[noitemsep,topsep=-5pt]
    \item Presented an open-source layout decomposition framework, with efficient implementations of various state-of-the-art simplification and decomposition algorithms.
    \item Compressed the model with up to 4.9$\times$ reduction of parameters at a cost of little loss.
\end{itemize}

\textit{Acceleration and Compression of DNNs \textbf{[{ICTAI'19, Best Student Paper Award}]}}
\begin{itemize}[noitemsep,topsep=-5pt]
    \item Proposed a unified framework to compress CNNs by combining both lowrankness and sparsity.
    \item Discovered a set of issues of previous algorithms and proposed corresponding solutions. \\
\end{itemize}

{\bf Research Assistant, Southern University of Science and Technology, China}  \hfill { June.~2018 -- Jan.~2019} \\ 
\textit{Testing of Auto-driving Systems \textbf{[{ICSE'20}]}}
\begin{itemize}[noitemsep,topsep=-5pt]
    \item Introduced a joint optimization method to systematically generate adversarial perturbations to mislead steering of an autonomous driving physically.
    \item The first study demonstrating the possibility of continuous physical-world tests for auto-driving scenarios.
\end{itemize}

\textit{Fault Localization \textbf{[{ISSTA'19, Distinguished Paper Award}]}}
\begin{itemize}[noitemsep,topsep=-5pt]
    \item Proposed a DL approach to automatically learn the most effective features for precise fault localization. 
    \item Significantly outperformed state-of-the-art with over 20\% improvement. \\
\end{itemize}
\end{rSection}
